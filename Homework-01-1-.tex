% Options for packages loaded elsewhere
\PassOptionsToPackage{unicode}{hyperref}
\PassOptionsToPackage{hyphens}{url}
%
\documentclass[
]{article}
\usepackage{lmodern}
\usepackage{amssymb,amsmath}
\usepackage{ifxetex,ifluatex}
\ifnum 0\ifxetex 1\fi\ifluatex 1\fi=0 % if pdftex
  \usepackage[T1]{fontenc}
  \usepackage[utf8]{inputenc}
  \usepackage{textcomp} % provide euro and other symbols
\else % if luatex or xetex
  \usepackage{unicode-math}
  \defaultfontfeatures{Scale=MatchLowercase}
  \defaultfontfeatures[\rmfamily]{Ligatures=TeX,Scale=1}
\fi
% Use upquote if available, for straight quotes in verbatim environments
\IfFileExists{upquote.sty}{\usepackage{upquote}}{}
\IfFileExists{microtype.sty}{% use microtype if available
  \usepackage[]{microtype}
  \UseMicrotypeSet[protrusion]{basicmath} % disable protrusion for tt fonts
}{}
\makeatletter
\@ifundefined{KOMAClassName}{% if non-KOMA class
  \IfFileExists{parskip.sty}{%
    \usepackage{parskip}
  }{% else
    \setlength{\parindent}{0pt}
    \setlength{\parskip}{6pt plus 2pt minus 1pt}}
}{% if KOMA class
  \KOMAoptions{parskip=half}}
\makeatother
\usepackage{xcolor}
\IfFileExists{xurl.sty}{\usepackage{xurl}}{} % add URL line breaks if available
\IfFileExists{bookmark.sty}{\usepackage{bookmark}}{\usepackage{hyperref}}
\hypersetup{
  pdftitle={Homework 1},
  hidelinks,
  pdfcreator={LaTeX via pandoc}}
\urlstyle{same} % disable monospaced font for URLs
\usepackage[margin=1in]{geometry}
\usepackage{color}
\usepackage{fancyvrb}
\newcommand{\VerbBar}{|}
\newcommand{\VERB}{\Verb[commandchars=\\\{\}]}
\DefineVerbatimEnvironment{Highlighting}{Verbatim}{commandchars=\\\{\}}
% Add ',fontsize=\small' for more characters per line
\usepackage{framed}
\definecolor{shadecolor}{RGB}{248,248,248}
\newenvironment{Shaded}{\begin{snugshade}}{\end{snugshade}}
\newcommand{\AlertTok}[1]{\textcolor[rgb]{0.94,0.16,0.16}{#1}}
\newcommand{\AnnotationTok}[1]{\textcolor[rgb]{0.56,0.35,0.01}{\textbf{\textit{#1}}}}
\newcommand{\AttributeTok}[1]{\textcolor[rgb]{0.77,0.63,0.00}{#1}}
\newcommand{\BaseNTok}[1]{\textcolor[rgb]{0.00,0.00,0.81}{#1}}
\newcommand{\BuiltInTok}[1]{#1}
\newcommand{\CharTok}[1]{\textcolor[rgb]{0.31,0.60,0.02}{#1}}
\newcommand{\CommentTok}[1]{\textcolor[rgb]{0.56,0.35,0.01}{\textit{#1}}}
\newcommand{\CommentVarTok}[1]{\textcolor[rgb]{0.56,0.35,0.01}{\textbf{\textit{#1}}}}
\newcommand{\ConstantTok}[1]{\textcolor[rgb]{0.00,0.00,0.00}{#1}}
\newcommand{\ControlFlowTok}[1]{\textcolor[rgb]{0.13,0.29,0.53}{\textbf{#1}}}
\newcommand{\DataTypeTok}[1]{\textcolor[rgb]{0.13,0.29,0.53}{#1}}
\newcommand{\DecValTok}[1]{\textcolor[rgb]{0.00,0.00,0.81}{#1}}
\newcommand{\DocumentationTok}[1]{\textcolor[rgb]{0.56,0.35,0.01}{\textbf{\textit{#1}}}}
\newcommand{\ErrorTok}[1]{\textcolor[rgb]{0.64,0.00,0.00}{\textbf{#1}}}
\newcommand{\ExtensionTok}[1]{#1}
\newcommand{\FloatTok}[1]{\textcolor[rgb]{0.00,0.00,0.81}{#1}}
\newcommand{\FunctionTok}[1]{\textcolor[rgb]{0.00,0.00,0.00}{#1}}
\newcommand{\ImportTok}[1]{#1}
\newcommand{\InformationTok}[1]{\textcolor[rgb]{0.56,0.35,0.01}{\textbf{\textit{#1}}}}
\newcommand{\KeywordTok}[1]{\textcolor[rgb]{0.13,0.29,0.53}{\textbf{#1}}}
\newcommand{\NormalTok}[1]{#1}
\newcommand{\OperatorTok}[1]{\textcolor[rgb]{0.81,0.36,0.00}{\textbf{#1}}}
\newcommand{\OtherTok}[1]{\textcolor[rgb]{0.56,0.35,0.01}{#1}}
\newcommand{\PreprocessorTok}[1]{\textcolor[rgb]{0.56,0.35,0.01}{\textit{#1}}}
\newcommand{\RegionMarkerTok}[1]{#1}
\newcommand{\SpecialCharTok}[1]{\textcolor[rgb]{0.00,0.00,0.00}{#1}}
\newcommand{\SpecialStringTok}[1]{\textcolor[rgb]{0.31,0.60,0.02}{#1}}
\newcommand{\StringTok}[1]{\textcolor[rgb]{0.31,0.60,0.02}{#1}}
\newcommand{\VariableTok}[1]{\textcolor[rgb]{0.00,0.00,0.00}{#1}}
\newcommand{\VerbatimStringTok}[1]{\textcolor[rgb]{0.31,0.60,0.02}{#1}}
\newcommand{\WarningTok}[1]{\textcolor[rgb]{0.56,0.35,0.01}{\textbf{\textit{#1}}}}
\usepackage{graphicx,grffile}
\makeatletter
\def\maxwidth{\ifdim\Gin@nat@width>\linewidth\linewidth\else\Gin@nat@width\fi}
\def\maxheight{\ifdim\Gin@nat@height>\textheight\textheight\else\Gin@nat@height\fi}
\makeatother
% Scale images if necessary, so that they will not overflow the page
% margins by default, and it is still possible to overwrite the defaults
% using explicit options in \includegraphics[width, height, ...]{}
\setkeys{Gin}{width=\maxwidth,height=\maxheight,keepaspectratio}
% Set default figure placement to htbp
\makeatletter
\def\fps@figure{htbp}
\makeatother
\setlength{\emergencystretch}{3em} % prevent overfull lines
\providecommand{\tightlist}{%
  \setlength{\itemsep}{0pt}\setlength{\parskip}{0pt}}
\setcounter{secnumdepth}{-\maxdimen} % remove section numbering

\title{Homework 1}
\author{}
\date{\vspace{-2.5em}}

\begin{document}
\maketitle

\begin{enumerate}
\def\labelenumi{\arabic{enumi}.}
\tightlist
\item
  The Iowa data set iowa.csv is a toy example that summarises the yield
  of wheat (bushels per acre) for the state of Iowa between 1930-1962.
  In addition to yield, year, rainfall and temperature were recorded as
  the main predictors of yield.

  \begin{enumerate}
  \def\labelenumii{\alph{enumii}.}
  \tightlist
  \item
    First, we need to load the data set into R using the command
    \texttt{read.csv()}. Use the help function to learn what arguments
    this function takes. Once you have the necessary input, load the
    data set into R and make it a data frame called \texttt{iowa.df}.
  \item
    How many rows and columns does \texttt{iowa.df} have?
  \item
    What are the names of the columns of \texttt{iowa.df}?
  \item
    What is the value of row 5, column 7 of \texttt{iowa.df}?
  \item
    Display the second row of \texttt{iowa.df} in its entirety.
  \end{enumerate}
\end{enumerate}

\begin{Shaded}
\begin{Highlighting}[]
\NormalTok{iowa.df<-}\KeywordTok{read.csv}\NormalTok{(}\StringTok{"data/iowa.csv"}\NormalTok{,}\DataTypeTok{header=}\NormalTok{T,}\DataTypeTok{sep=}\StringTok{";"}\NormalTok{)}
\NormalTok{nc <-}\StringTok{ }\KeywordTok{ncol}\NormalTok{(iowa.df);nc}
\end{Highlighting}
\end{Shaded}

\begin{verbatim}
## [1] 10
\end{verbatim}

\begin{Shaded}
\begin{Highlighting}[]
\NormalTok{nr <-}\StringTok{ }\KeywordTok{nrow}\NormalTok{(iowa.df);nr}
\end{Highlighting}
\end{Shaded}

\begin{verbatim}
## [1] 33
\end{verbatim}

\begin{Shaded}
\begin{Highlighting}[]
\NormalTok{iowa.df[}\DecValTok{5}\NormalTok{,}\DecValTok{7}\NormalTok{]}
\end{Highlighting}
\end{Shaded}

\begin{verbatim}
## [1] 79.7
\end{verbatim}

\begin{Shaded}
\begin{Highlighting}[]
\NormalTok{iowa.df[}\DecValTok{2}\NormalTok{,]}
\end{Highlighting}
\end{Shaded}

\begin{verbatim}
##   Year Rain0 Temp1 Rain1 Temp2 Rain2 Temp3 Rain3 Temp4 Yield
## 2 1931 14.76  57.5  3.83    75  2.72  77.2   3.3  72.6  32.9
\end{verbatim}

\begin{enumerate}
\def\labelenumi{\arabic{enumi}.}
\setcounter{enumi}{1}
\tightlist
\item
  Syntax and class-typing.

  \begin{enumerate}
  \def\labelenumii{\alph{enumii}.}
  \tightlist
  \item
    For each of the following commands, either explain why they should
    be errors, or explain the non-erroneous result.
  \end{enumerate}
\end{enumerate}

\begin{verbatim}
vector1 <- c("5", "12", "7", "32")
max(vector1)
sort(vector1)
sum(vector1)
\end{verbatim}

vector1
的元素是字符(character),max()返回字符串中第一个字符对应顺序最大的字符,sort()也是按这个规则进行从小到大的排列,而sum()函数对象是数字不是字符。
b. For the next series of commands, either explain their results, or why
they should produce errors.

\begin{verbatim}
vector2 <- c("5",7,12)
vector2[2] + vector2[3]

dataframe3 <- data.frame(z1="5",z2=7,z3=12)
dataframe3[1,2] + dataframe3[1,3]

list4 <- list(z1="6", z2=42, z3="49", z4=126)
list4[[2]]+list4[[4]]
list4[2]+list4[4]
\end{verbatim}

\begin{enumerate}
\def\labelenumi{\arabic{enumi}.}
\setcounter{enumi}{2}
\tightlist
\item
  Working with functions and operators.

  \begin{enumerate}
  \def\labelenumii{\alph{enumii}.}
  \tightlist
  \item
    The colon operator will create a sequence of integers in order. It
    is a special case of the function \texttt{seq()} which you saw
    earlier in this assignment. Using the help command \texttt{?seq} to
    learn about the function, design an expression that will give you
    the sequence of numbers from 1 to 10000 in increments of 372. Design
    another that will give you a sequence between 1 and 10000 that is
    exactly 50 numbers in length.
  \item
    The function \texttt{rep()} repeats a vector some number of times.
    Explain the difference between `rep(1:3, times=3) and rep(1:3,
    each=3).
  \end{enumerate}
\end{enumerate}

MB.Ch1.2. The orings data frame gives data on the damage that had
occurred in US space shuttle launches prior to the disastrous Challenger
launch of 28 January 1986. The observations in rows 1, 2, 4, 11, 13, and
18 were included in the pre-launch charts used in deciding whether to
proceed with the launch, while remaining rows were omitted.

Create a new data frame by extracting these rows from orings, and plot
total incidents against temperature for this new data frame. Obtain a
similar plot for the full data set.

MB.Ch1.4. For the data frame ais (DAAG package)

\begin{enumerate}
\def\labelenumi{(\alph{enumi})}
\item
  Use the function str() to get information on each of the columns.
  Determine whether any of the columns hold missing values.
\item
  Make a table that shows the numbers of males and females for each
  different sport. In which sports is there a large imbalance (e.g., by
  a factor of more than 2:1) in the numbers of the two sexes?
\end{enumerate}

MB.Ch1.6.Create a data frame called Manitoba.lakes that contains the
lake's elevation (in meters above sea level) and area (in square
kilometers) as listed below. Assign the names of the lakes using the
row.names() function. elevation area Winnipeg 217 24387 Winnipegosis 254
5374 Manitoba 248 4624 SouthernIndian 254 2247 Cedar 253 1353 Island 227
1223 Gods 178 1151 Cross 207 755 Playgreen 217 657

\begin{enumerate}
\def\labelenumi{(\alph{enumi})}
\tightlist
\item
  Use the following code to plot log2(area) versus elevation, adding
  labeling infor- mation (there is an extreme value of area that makes a
  logarithmic scale pretty much essential):
\end{enumerate}

\begin{Shaded}
\begin{Highlighting}[]
\KeywordTok{attach}\NormalTok{(Manitoba.lakes)}
\KeywordTok{plot}\NormalTok{(}\KeywordTok{log2}\NormalTok{(area) }\OperatorTok{~}\StringTok{ }\NormalTok{elevation, }\DataTypeTok{pch=}\DecValTok{16}\NormalTok{, }\DataTypeTok{xlim=}\KeywordTok{c}\NormalTok{(}\DecValTok{170}\NormalTok{,}\DecValTok{280}\NormalTok{))}
\CommentTok{# NB: Doubling the area increases log2(area) by 1.0}
\KeywordTok{text}\NormalTok{(}\KeywordTok{log2}\NormalTok{(area) }\OperatorTok{~}\StringTok{ }\NormalTok{elevation, }\DataTypeTok{labels=}\KeywordTok{row.names}\NormalTok{(Manitoba.lakes), }\DataTypeTok{pos=}\DecValTok{4}\NormalTok{)}
\KeywordTok{text}\NormalTok{(}\KeywordTok{log2}\NormalTok{(area) }\OperatorTok{~}\StringTok{ }\NormalTok{elevation, }\DataTypeTok{labels=}\NormalTok{area, }\DataTypeTok{pos=}\DecValTok{2}\NormalTok{) }
\KeywordTok{title}\NormalTok{(}\StringTok{"Manitoba’s Largest Lakes"}\NormalTok{)}
\end{Highlighting}
\end{Shaded}

\includegraphics{Homework-01-1-_files/figure-latex/unnamed-chunk-7-1.pdf}
Devise captions that explain the labeling on the points and on the
y-axis. It will be necessary to explain how distances on the scale
relate to changes in area.

\begin{enumerate}
\def\labelenumi{(\alph{enumi})}
\setcounter{enumi}{1}
\tightlist
\item
  Repeat the plot and associated labeling, now plotting area versus
  elevation, but specifying log=``y'' in order to obtain a logarithmic
  y-scale.
\end{enumerate}

\begin{Shaded}
\begin{Highlighting}[]
\KeywordTok{plot}\NormalTok{(area }\OperatorTok{~}\StringTok{ }\NormalTok{elevation, }\DataTypeTok{pch=}\DecValTok{16}\NormalTok{, }\DataTypeTok{xlim=}\KeywordTok{c}\NormalTok{(}\DecValTok{170}\NormalTok{,}\DecValTok{280}\NormalTok{), }\DataTypeTok{ylog=}\NormalTok{T)}
\KeywordTok{text}\NormalTok{(area }\OperatorTok{~}\StringTok{ }\NormalTok{elevation, }\DataTypeTok{labels=}\KeywordTok{row.names}\NormalTok{(Manitoba.lakes), }\DataTypeTok{pos=}\DecValTok{4}\NormalTok{, }\DataTypeTok{ylog=}\NormalTok{T)}
\KeywordTok{text}\NormalTok{(area }\OperatorTok{~}\StringTok{ }\NormalTok{elevation, }\DataTypeTok{labels=}\NormalTok{area, }\DataTypeTok{pos=}\DecValTok{2}\NormalTok{, }\DataTypeTok{ylog=}\NormalTok{T) }
\KeywordTok{title}\NormalTok{(}\StringTok{"Manitoba’s Largest Lakes"}\NormalTok{)}
\end{Highlighting}
\end{Shaded}

\includegraphics{Homework-01-1-_files/figure-latex/unnamed-chunk-8-1.pdf}
MB.Ch1.7. Look up the help page for the R function dotchart(). Use this
function to display the areas of the Manitoba lakes (a) on a linear
scale, and (b) on a logarithmic scale. Add, in each case, suitable
labeling information.

\begin{Shaded}
\begin{Highlighting}[]
\KeywordTok{dotchart}\NormalTok{(}\KeywordTok{log2}\NormalTok{(area))}
\end{Highlighting}
\end{Shaded}

\includegraphics{Homework-01-1-_files/figure-latex/unnamed-chunk-9-1.pdf}

MB.Ch1.8. Using the sum() function, obtain a lower bound for the area of
Manitoba covered by water.

\end{document}
